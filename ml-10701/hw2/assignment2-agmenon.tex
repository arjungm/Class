%%%%%%%%%%%%%%%%%%%%%%%%%%%%%%%%%%%%%%%%%
% Thin Sectioned Essay
% LaTeX Template
% Version 1.0 (3/8/13)
%
% This template has been downloaded from:
% http://www.LaTeXTemplates.com
%
% Original Author:
% Nicolas Diaz (nsdiaz@uc.cl) with extensive modifications by:
% Vel (vel@latextemplates.com)
%
% License:
% CC BY-NC-SA 3.0 (http://creativecommons.org/licenses/by-nc-sa/3.0/)
%
%%%%%%%%%%%%%%%%%%%%%%%%%%%%%%%%%%%%%%%%%

%----------------------------------------------------------------------------------------
%	PACKAGES AND OTHER DOCUMENT CONFIGURATIONS
%----------------------------------------------------------------------------------------

\documentclass[letterpaper,10pt]{article} % Font size (can be 10pt, 11pt or 12pt) and paper size (remove a4paper for US letter paper)
\usepackage{geometry}
\geometry{paperwidth=4in,margin=.1in,paperheight=30in}
\usepackage[protrusion=true,expansion=true]{microtype} % Better typography
\usepackage{graphicx} % Required for including pictures
\usepackage{caption}
\usepackage{subcaption}
\usepackage{wrapfig} % Allows in-line images
\usepackage{amsmath,amsthm}
\usepackage{thmtools}
\usepackage{mathpazo} % Use the Palatino font
\usepackage[T1]{fontenc} % Required for accented characters
\linespread{1.05} % Change line spacing here, Palatino benefits from a slight increase by default

\makeatletter
\newcommand{\pipe}{\;\middle\vert\;}
\newcommand{\condp}[2]{\Pr\left( #1 \pipe #2 \right)}
\newcommand{\pr}[1]{\Pr\left( #1 \right)}
\newcommand{\condpe}[2]{\frac{\pr{#1,#2}}{\pr{#2}}}
\newcommand{\condpb}[2]{\frac{\condp{#2}{#1}\pr{#1}}{\pr{#2}}}
\newcommand{\norm}[1]{\left|#1\right|}
\DeclareMathOperator*{\argmin}{arg\,min}
\DeclareMathOperator*{\argmax}{arg\,max}
%theorem styling
\declaretheorem{theorem} 
\declaretheoremstyle[%
spaceabove=-6pt,%
spacebelow=6pt,%
headfont=\normalfont\itshape,%
postheadspace=1em,%
qed=\qedsymbol,%
headpunct={}
]{mystyle} 
\declaretheorem[name={},style=mystyle,unnumbered,
]{Proof}

\newcommand{\prove}[1]{
\begin{Proof}
\begin{align*}
#1
\end{align*}
\end{Proof}
}

\newcommand{\lm}{\lambda}
\newcommand{\LML}[2]{\lm #1 + (1-\lm) #2}

\renewcommand{\@listI}{\itemsep=0pt} % Reduce the space between items in the itemize and enumerate environments and the bibliography

\renewcommand{\maketitle}{ % Customize the title - do not edit title and author name here, see the TITLE block below
\begin{flushright} % Right align
{\LARGE\@title} % Increase the font size of the title

{\large\@author} % Author name
\\\@date % Date

\end{flushright}
}

%----------------------------------------------------------------------------------------
%	TITLE
%----------------------------------------------------------------------------------------

\title{\textbf{Assignment 2}\\ % Title
Machine Learning 10-701} % Subtitle

\author{\textsc{Arjun Menon} % Author
\\{\textit{Carnegie Mellon University}}} % Institution

%----------------------------------------------------------------------------------------

\begin{document}

\maketitle % Print the title section

\section{Convexity}
\subsection{}
\subsubsection*{(a)}
\begin{align*}
&h_1(x) = f(Ax+b)\\
\end{align*}
For convexity,
\begin{align*}
&\lm f(Ax+b) + (1-\lm)f(Ay+b) \geq f(A(\lm x + (1-\lm)y) +b)\\
\end{align*}
Let $\alpha = Ax+b$ and $\beta = Ay+b$,
\begin{align*}
\lm f(\alpha) + (1-\lm) f(\beta) \geq f(\lm(\alpha) + (1-\lm)\beta)\\
\end{align*}
Which is true.

\subsubsection*{(b)}
\begin{align*}
h_2(\LML{x}{y}) \leq \lm h_2(x) + (1-\lm)h_2(y) \\
\end{align*}
Decomposes to two inequalities,
\begin{align*}
f(\LML{x}{y}) \leq h_2(x) + (1-\lm)h_2(y)\\
\end{align*}
and\\
\begin{align*}
g(\LML{x}{y}) \leq \lm h_2(x) + (1-\lm)h_2(y)\\
\end{align*}
If $f$ is strictly greater than $g$ everywhere, then $h_2(x)$ is trivially convex. However at the end points $x$ and $y$, if $g$ is greater then it follows that for one of these cases that,\\
\begin{align*}
f(\LML{x}{y}) &\leq \LML{f(x)}{f(y)}\\
          &\leq \LML{g(x)}{f(y)}\\
          &\leq \lm h_2(x) + (1-\lm)h_2(y) \\
\end{align*}
Which shows convexity of $h_2$ holds, if you permute for f or g being greater at either end points since lines are convex functions.

\subsubsection*{(c)}
\begin{align*}
\LML{h_3(x)}{h_3(y)} &\geq h_3(\LML{x}{y})\\
&\geq g*f(\LML{x}{y})\\
\LML{g*f(x)}{g*f(y)} &\geq g*f(\LML{x}{y})\\
\end{align*}
Substituting $f(x)=\alpha$ and $g(x)=\beta$,
\begin{align*}
\LML{g(\alpha)}{g(\beta)} \geq g(\LML{\alpha}{\beta})\\
\end{align*}
Which is true.
\subsection{}

\begin{align*}
f(x+\lm(y-x))-f(x) &\leq \lm(f(y)-f(x))\\
(y-x)\lim_{\lm\to 0} \frac{f(x+\lm(y-x))-f(x)}{\lm(y-x)} &\leq f(y) - f(x)\\
(y-x)f'(x) &\leq f(y) - f(x)\\
\end{align*}

\subsection{}
\subsubsection*{(a)}
If,
\begin{align*}
f(\LML{x}{y} &\leq \LML{f(x)}{f(y)} - \frac{1}{2} n \lm (1 - \lm) \left|x-y\right|^2_2\\
&\leq \LML{f(x)}{f(y)} - \frac{1}{2} m \lm (1 - \lm) \left|x-y\right|^2_2\\
\end{align*}

Then,
\begin{align*}
- \frac{1}{2} n \lm (1 - \lm) \left|x-y\right|^2_2 \leq - \frac{1}{2} m \lm (1 - \lm) \left|x-y\right|^2_2\\
\end{align*}

can only be true for $m > n$, for $\lm \in (0,1)$ and $x,y \in$ the domain of $f$

\subsubsection*{(b)}
Given,
\begin{align*}
f(\LML{x}{y}) \leq \LML{f(x)}{f(y)} - \frac{1}{2} m \lm (1 - \lm) \left|x-y\right|^2_2\\
\end{align*}
Then for strict convexity, we want that,
\begin{align*}
\frac{1}{2} m \lm (1 - \lm) \left|x-y\right|^2_2 > 0
\end{align*}
Which is guaranteed by the following constituent terms being nonzero, when $\lm \in (0,1)$ and $x\ne y$:
\begin{itemize}
\item $m\ne0$, for strong convexity
\item $\lm\ne 0$, condition for strict
\item $\lm\ne 1$, condition for strict
\item $\left|x-y\right|^2_2 \ne 0$, condition for strict
\end{itemize}

Thus strong convexity implies strict convexity.
\subsubsection*{(c)}
\begin{align*}
f(x) = x^4\\
f'(x) = 4x^3\\
f''(x) = 12x^2\\
\end{align*}

The second order differential shows that $f''(x) > 0$ for all $x \in  (0,\infty]$ which gives us our strict convexity domain.

Note however that there is no $m$ for $12x^2 \geq m$, because for any $m$ you can get a lower $x$ that violates this bound on the domain.
\subsection{}
\subsubsection*{(a)}
\begin{align*}
f(x) = x^2 + x^4
f'(x) = 2x + 4x^3
f''(x) = 2 + 12x^2 > 0
\end{align*}

for all $x \in [-\infty,\infty]$ which makes it strongly convex for $x=2$.

\subsubsection*{(b)}
\begin{align*}
f(x) = x^2 + x^4
f'(x) = 2x + 4x^3
f''(x) = 2 + 12x^2 \geq 2 + 12 1^2 = 14
\end{align*}
Which proves it is strongly convex for $m=14$.

\subsubsection*{(c)}
Given,
\begin{align*}
x,y \in \mathbb{R}^d, \lm \in (0,1)
\end{align*}
We have by triangle inequality of the norms, and linearity,
\begin{align*}
\left| \LML{x}{y}\right|^2 &\leq \left|\lm x\right| + \left|(1-\lm)y\right|
&= \lm\left| x\right| + (1-\lm)\left|y\right|
\end{align*}

Which is convex, but not strictly convex if you just take the example where $y = a x$.

%------------------------------------------------
%\bibliographystyle{unsrt}
%\bibliography{sample}

%----------------------------------------------------------------------------------------

\end{document}
