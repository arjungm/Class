%%%%%%%%%%%%%%%%%%%%%%%%%%%%%%%%%%%%%%%%%
% Thin Sectioned Essay
% LaTeX Template
% Version 1.0 (3/8/13)
%
% This template has been downloaded from:
% http://www.LaTeXTemplates.com
%
% Original Author:
% Nicolas Diaz (nsdiaz@uc.cl) with extensive modifications by:
% Vel (vel@latextemplates.com)
%
% License:
% CC BY-NC-SA 3.0 (http://creativecommons.org/licenses/by-nc-sa/3.0/)
%
%%%%%%%%%%%%%%%%%%%%%%%%%%%%%%%%%%%%%%%%%

%----------------------------------------------------------------------------------------
%	PACKAGES AND OTHER DOCUMENT CONFIGURATIONS
%----------------------------------------------------------------------------------------

\documentclass[letterpaper,10pt]{article} % Font size (can be 10pt, 11pt or 12pt) and paper size (remove a4paper for US letter paper)
\usepackage[margin=1.6in]{geometry}
\usepackage[protrusion=true,expansion=true]{microtype} % Better typography
\usepackage{graphicx} % Required for including pictures
\usepackage{wrapfig} % Allows in-line images
\usepackage{amsmath,amsthm}
\usepackage{thmtools}
\usepackage{mathpazo} % Use the Palatino font
\usepackage[T1]{fontenc} % Required for accented characters
\linespread{1.05} % Change line spacing here, Palatino benefits from a slight increase by default

\makeatletter
\newcommand{\pipe}{\;\middle\vert\;}
\newcommand{\condp}[2]{\Pr\left( #1 \pipe #2 \right)}
\newcommand{\pr}[1]{\Pr\left( #1 \right)}
\newcommand{\condpe}[2]{\frac{\pr{#1,#2}}{\pr{#2}}}
\newcommand{\norm}[1]{\left|#1\right|}

%theorem styling
\declaretheorem{theorem} 
\declaretheoremstyle[%
spaceabove=-6pt,%
spacebelow=6pt,%
headfont=\normalfont\itshape,%
postheadspace=1em,%
qed=\qedsymbol,%
headpunct={}
]{mystyle} 
\declaretheorem[name={},style=mystyle,unnumbered,
]{Proof}

\newcommand{\prove}[1]{
\begin{Proof}
\begin{align*}
#1
\end{align*}
\end{Proof}
}

\renewcommand{\@listI}{\itemsep=0pt} % Reduce the space between items in the itemize and enumerate environments and the bibliography

\renewcommand{\maketitle}{ % Customize the title - do not edit title and author name here, see the TITLE block below
\begin{flushright} % Right align
{\LARGE\@title} % Increase the font size of the title

{\large\@author} % Author name
\\\@date % Date

\end{flushright}
}

%----------------------------------------------------------------------------------------
%	TITLE
%----------------------------------------------------------------------------------------

\title{\textbf{Assignment 1}\\ % Title
Machine Learning 10-701} % Subtitle

\author{\textsc{Arjun Menon} % Author
\\{\textit{Carnegie Mellon University}}} % Institution

%----------------------------------------------------------------------------------------

\begin{document}

\maketitle % Print the title section

\section{Probability Review [Ahmed]}

\subsection{Why just 2 variables? Let's go for 3}

\subsubsection{}\label{subsubsec:1}
From the law of conditional probability,

\prove{
\frac{\condp{A,B}{C}}{\condp{B}{C}} &= \frac{\pr{A,B,C}}{\pr{C}} \frac{\pr{C}}{\pr{B,C}}\\
                                    &= \frac{\pr{A,B,C}}{\pr{B,C}}\\
                                    &= \frac{\pr{A,D}}{\pr{D}}\\
                                    &= \condp{A}{D}\\
                                    &= \condp{A}{B,C} \qedhere
}

\subsubsection{}\label{subsubsec:2}
\prove{
\sum_{B}\condp{A,B}{C} &= \sum_{B}\condpe{A,B}{C}\\
&=\condpe{A}{C}\\
&=\condp{A}{C} \qedhere
}

\subsubsection{}
Using result from Problem~\ref{subsubsec:1} and Problem~\ref{subsubsec:2},
\prove{
\sum_{B}\condp{A}{B,C}\condp{B}{C} &= \sum_{B}\condp{A,B}{C}\\
&= \condp{A}{C} \qedhere
}

\subsection{Evaluating Test Results}

\subsubsection{}
The probability that a transation succeeds given that it was handled by $A2$ is
\begin{align*}
\condp{Success}{A=2} &= \condpe{Success}{A=2}\\
&= \frac{\norm{Success, A=2}}{\norm{A=2}}\\
&= \frac{2150}{2150+500}\\
&= 0.811
\end{align*}

\subsubsection{}
If we recommend $A2$ then we need to see that
\prove{
\condp{Success}{A=2} &\geq \condp{Success}{A=1}\\
0.811 &\geq \frac{6000}{6000+1700}\\
0.811 &\geq 0.779 \qedhere
}

\subsubsection{}
The statement about the probability of $A2$ handling a transaction successfully given $A1$ handled it successfully is given by,
\prove{
\condp{A2_{success}=1}{A1_{success}=1} &= \condpe{A2_{success}=1}{A1_{success}=1}\\
&\geq \frac{\pr{A2_{success}=1}+\pr{A1_{success}=1}-1}{\pr{A1_{success}=1}}\\
&= \frac{\frac{2150}{2150+500}+\frac{6000}{6000+1700}-1}{\frac{6000}{6000+1700}}\\
&= 0.757 \\
&\not\leq 0.70 \qedhere
}

\subsection{Monty Hall Problem}

Solving for $\pr{car3 \pipe open2, choose1}$ we get,
\begin{align*}
 \pr{car3 \pipe open2, choose1} &= \frac{\condp{car3, open2}{choose1}}{\condp{open2}{choose1}}\\
 &=\frac{\condp{open2}{choose1,car3}\condp{car3}{choose1}}{\condp{open2}{choose1}} \\
 &=\frac{1 \times \tfrac{1}{3}}{\tfrac{1}{2}} \\
 &=\tfrac{2}{3}
\end{align*}


and solving for $\pr{car1 \pipe open2, choose1}$ we get,
\begin{align*}
 \pr{car1 \pipe open2, choose1} &= \frac{\condp{car1, open2}{choose1}}{\condp{open2}{choose1}}\\
 &=\frac{\condp{open2}{choose1,car1}\condp{car1}{choose1}}{\condp{open2}{choose1}} \\
 &=\frac{\tfrac{1}{2} \times \tfrac{1}{3}}{\tfrac{1}{2}} \\
 &=\tfrac{1}{3}
\end{align*}

which gives us,

\prove{
  \frac{\pr{car3 \pipe open2, choose1}}{\pr{car1 \pipe open2, choose1}} &= \frac{\tfrac{2}{3}}{\tfrac{1}{3}} \\
  &= 2 \qedhere
}

\newpage
\section{Regression [Leila]}
\subsection{Linear Regression}
\subsection{Ridge Regression}
\newpage
\section{Classification [Dougal]}
\subsection{Drawing decision boundaries}
\subsection{Defeating classifiers}

\newpage
\section{Coding Competition [Carlton]}


%------------------------------------------------
%\bibliographystyle{unsrt}
%\bibliography{sample}

%----------------------------------------------------------------------------------------

\end{document}
