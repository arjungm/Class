%%%%%%%%%%%%%%%%%%%%%%%%%%%%%%%%%%%%%%%%%
% Thin Sectioned Essay
% LaTeX Template
% Version 1.0 (3/8/13)
%
% This template has been downloaded from:
% http://www.LaTeXTemplates.com
%
% Original Author:
% Nicolas Diaz (nsdiaz@uc.cl) with extensive modifications by:
% Vel (vel@latextemplates.com)
%
% License:
% CC BY-NC-SA 3.0 (http://creativecommons.org/licenses/by-nc-sa/3.0/)
%
%%%%%%%%%%%%%%%%%%%%%%%%%%%%%%%%%%%%%%%%%

%----------------------------------------------------------------------------------------
%	PACKAGES AND OTHER DOCUMENT CONFIGURATIONS
%----------------------------------------------------------------------------------------

\documentclass[letterpaper,10pt]{article} % Font size (can be 10pt, 11pt or 12pt) and paper size (remove a4paper for US letter paper)
\usepackage[margin=1.6in]{geometry}
\usepackage[protrusion=true,expansion=true]{microtype} % Better typography
\usepackage{graphicx} % Required for including pictures
\usepackage{wrapfig} % Allows in-line images

\usepackage{mathpazo} % Use the Palatino font
\usepackage[T1]{fontenc} % Required for accented characters
\linespread{1.05} % Change line spacing here, Palatino benefits from a slight increase by default

\makeatletter
\renewcommand\@biblabel[1]{\textbf{#1.}} % Change the square brackets for each bibliography item from '[1]' to '1.'
\renewcommand{\@listI}{\itemsep=0pt} % Reduce the space between items in the itemize and enumerate environments and the bibliography

\renewcommand{\maketitle}{ % Customize the title - do not edit title and author name here, see the TITLE block below
\begin{flushright} % Right align
{\LARGE\@title} % Increase the font size of the title

\vspace{5pt} % Some vertical space between the title and author name

{\large\@author} % Author name
\vspace{0pt} % Some vertical space between the author block and abstract
\end{flushright}
}

%----------------------------------------------------------------------------------------
%	TITLE
%----------------------------------------------------------------------------------------

\title{\textbf{Manipulation Algorithms Project Proposal}\\ % Title
Trajectory Retiming for Manipulation Planning} % Subtitle

\author{\textsc{Arjun Menon} % Author
\\{\textit{Carnegie Mellon University}}} % Institution

%----------------------------------------------------------------------------------------

\begin{document}

\maketitle % Print the title section

%----------------------------------------------------------------------------------------
%	ABSTRACT AND KEYWORDS
%----------------------------------------------------------------------------------------

%\renewcommand{\abstractname}{Summary} % Uncomment to change the name of the abstract to something else

%\begin{abstract}
%
%\end{abstract}
%
%\hspace*{3,6mm}\textit{Keywords:} lorem , ipsum , dolor , sit amet , lectus % Keywords
%
%\vspace{10pt} % Some vertical space between the abstract and first section

%----------------------------------------------------------------------------------------
%	ESSAY BODY
%----------------------------------------------------------------------------------------

\section{Problem}

This project is to design a motion planner that accounts for kinodynamic constraints and moving obstacles. The project aims to circumvent the curse of dimensionality in the kinodynamic configuration-space, $\mathbb{C}_d$, for search-based planners through adaptation of Bobrow's trajectory retiming algorithm~\cite{bobrow1985time}.

\section{Approach}\label{sec:approach}

\subsection{Kinodynamic Constraints}\label{sec:constraints}

Since planning in the kinematic configuration-space, $\mathbb{C}$, and consequently retiming the trajectory for dynamics is not a complete approach, the project will leverage previous work that allows us to efficiently assess whether a given path in $\mathbb{C}$ can be traversed with respect to the kinodynamic constraints~\cite{pham2013velocity}. By propogating the velocity intervals (the set of valid velocities) for a given waypoint in $\mathbb{C}$, the proposed planner is able to ensure that a valid retiming exists for a plan, that respects the kinodynamic constraints.

\subsection{Moving Obstacles}\label{subsec:movobs}
Dealing with moving obstacles may create inadmissible islands to retimings below the maximum-velocity curve of Bobrow's algorithm~\cite{shin1985minimum}. Literature survey currently does not turn up any method to path around these regions in the phase-plane, so the possible approaches the project will explore are:

\begin{itemize}
\item Graph-based search in the phase-plane using the velocity field over the phase plane as defining the motion primitives.
\item Integrating the SIPP framework~\cite{phillips2011sipp} with Bobrow's algorithm providing a way to compute the \emph{safe} time-intervals of the states in traversals.
\item Designing the switching point strategy for object collisions.
\end{itemize}

%------------------------------------------------

\section{Motivation}

A common theme among all of the common motion planning tasks for a robot arm is that the arm is manipulating objects in static environments. The manipulation of static objects is well researched and multiple approaches have been presented in the last decade  that are fast and proven to be reliable. However, autonomous manipulation amidst dynamic environments poses new challenges to motion planning. Sampling-based planners do well in kinodynamic spaces~\cite{lavalle2001randomized}, but search-based planners are intrinsically hamstringed by the high dimensional search-spaces. This project seeks to use Bobrow's algorithm as a means of dimensionality reduction, to enable tractable search-based kinodynamic motion planning.

%------------------------------------------------

\section{Deliverables}

\begin{tabular}{|p{0.1\columnwidth}|p{0.80\columnwidth}|p{0.10\columnwidth}|}
\hline
\textbf{Progress} & \textbf{Deliverables} & \textbf{When}\\
\hline
50\% & 
\begin{minipage}{0.80\columnwidth}
\vskip 4pt
\begin{itemize}
\item Implement and test numerically robust Bobrow's Algorithm.
\item Visualize collision regions in the Phase-plane with moving obstacles.
\item Develop augmented $\mathbb{C}$-space planner with velocity interval propogation.
\end{itemize}
\vskip 4pt
\end{minipage} &
Oct-30
\\
\hline
75\% & Develop and implement a scheme mentioned in Section~\ref{subsec:movobs}.&Nov-14\\
\hline
100\% & 
\begin{minipage}{0.80\columnwidth}
\vskip 4pt
\begin{itemize}
\item Develop naive approach using \emph{VIP} and simple collision checking.
\item Create test scenarios.
\item Compare naive approach with approach from Section~\ref{subsec:movobs}.
\item Analysis of planner variants.
\end{itemize}
\vskip 4pt
\end{minipage}&
Nov-28
\\
\hline
125\% &Try other approaches itemized in Section~\ref{subsec:movobs}.&Stretch
\\
\hline
\end{tabular}

%----------------------------------------------------------------------------------------
%	BIBLIOGRAPHY
%----------------------------------------------------------------------------------------

\bibliographystyle{unsrt}

\bibliography{bibliography}

%----------------------------------------------------------------------------------------

\end{document}
